\documentclass[a4paper,12pt]{article}

\usepackage[utf8]{inputenc}
\usepackage[T1]{fontenc} % per i caratteri accentati corretti in PDF
\usepackage{lmodern}
\usepackage[italian]{babel}
\renewcommand{\rmdefault}{lmss}
\usepackage{float}
\usepackage{geometry}
\usepackage{setspace}
\usepackage{enumitem}
\usepackage{titlesec}
\usepackage{tocloft}
\usepackage{graphicx}
\usepackage{hyperref}
\hypersetup{
    colorlinks=true,
    linkcolor=black,
    filecolor=magenta,      
    urlcolor=cyan,
}


\renewcommand{\contentsname}{Indice}

\geometry{margin=2.5cm}
\setstretch{1.2}

\titleformat{\section}{\large\bfseries}{\thesection}{1em}{}
\titleformat{\subsection}{\mdseries\bfseries}{\thesubsection}{1em}{}

\begin{document}

\begin{center}
    \small\hspace{10cm} 7zpus.swe@gmail.com\\
    \vspace{0.5cm}
    \Large \textbf{Analisi dei Capitolati di Progetto 2025/2026}\\
\end{center}

\vspace{0.3cm}
\hrule
\vspace{0.5cm}

\tableofcontents

\newpage

\section*{Tabella di Versionamento}
\begin{tabular}{|c|c|c|c|}
    \hline
    \textbf{Versione} & \textbf{Data} & \textbf{Autore}  & \textbf{Descrizione}            \\
    \hline
    1.8.1             & 30/10/2025    & Rocco Matteo A.  & Correzione refusi                         \\
    \hline
    1.8               & 27/10/2025    & Fattoni Antonio  & Aggiunta analisi capitolato C6                      \\
    \hline
    1.7               & 27/10/2025    & Gingillino Aaron & Aggiunta analisi capitolato C1                      \\
    \hline
    1.6               & 27/10/2025    & Georgescu Diana  & Aggiunta analisi capitolato C9                 \\
    \hline
    1.5               & 27/10/2025    & Vigolo Davide    & Aggiunta analisi capitolato C2                      \\
    \hline
    1.4.1             & 26/10/2025    & Rocco Matteo A.  & Correzione capitolato C5                   \\
    \hline
    1.4               & 26/10/2025    & Soligo Lorenzo   & Aggiunta analisi capitolato C8                      \\
    \hline
    1.3               & 26/10/2025    & Laoud Zakaria    & Aggiunta analisi capitolato C7                      \\
    \hline
    1.2               & 26/10/2025    & Rocco Matteo A.  & Aggiunta analisi capitolato C5                     \\
    \hline
    1.1               & 25/10/2025    & Georgescu Diana  & Aggiunta analisi capitolato C4          \\
    \hline
    1.0               & 17/10/2025    & Soligo Lorenzo   & \begin{tabular}[c]{@{}c@{}} Creazione documento e stesura iniziale\\ Analisi capitolato C3 \end{tabular} \\
    \hline

\end{tabular}

\section{Introduzione}
In questo documento sono analizzati i 9 Capitolati di Progetto proposti per
l`anno accademico 2025/2026, con l`obiettivo di valutarne la complessità, i
rischi associati e le potenzialità in termini di apprendimento e sviluppo delle
competenze del gruppo di progetto 7-ZPUs. Con alcune di queste aziende sono già
stati presi contatti preliminari (ERGON e Sanmarco Informatica). I verbali di
tali incontri sono presenti nella sezione
\href{https://github.com/7-ZPUs/Docs/tree/main/1_Candidatura/Verbali/Verbali\%20Esterni}{\textit{Verbali/VerbaliEsterni}} della repository di documentazione. 
I capitolati sono presentati in ordine di indice (C1-C9), tolto C3 che viene analizzato per primo in quanto capitolato scelto.

\section{Analisi del Capitolato scelto - C3 DIPReader}
\subsubsection*{Azienda Proponente:} Sanmarco Informatica S.p.A.
\subsubsection*{Committente:} Prof. Tullio Verdanega e Prof. Riccardo Cardin.
\subsubsection*{Obiettivo:} Sviluppo di un software per la lettura e ricerca di documenti digitali in formato .ZIP, con funzionalità avanzate di ricerca, verifica dell`autenticità. É inoltre necessario consultare i documenti all`interno del DIP permettendone il salvataggio in loco e la rielaborazione.
\subsubsection*{Dominio Applicativo:} L`azienda ha posto enfasi sul fatto che tale strumento risulterebbe molto utile in ambiti legali e giudiziari in cui vi è la necessità di cercare documenti specifici all`interno di una mole importante di dati, garantendone l`integrità, l`autenticità, la leggibilità  e la reperibilità.
\subsubsection*{Dominio Tecnico:}
\begin{itemize}
    \item \textbf{Database}: SQLite e/o FAISS. Il primo potrebbe essere scelto per rappresentare relazioni tra i documenti, mantenendo portabile la struttura del DIPReader. FAISS, invece, potrebbe essere utilizzato per implementare funzionalità di ricerca basate su similarità e campi semantici.
    \item \textbf{Framework Frontend}: Angular o React, entrambi validi e sostanzialmente equivalenti. Consigliato però l`uso di TypeScript per una maggiore robustezza del codice.
    \item \textbf{Strumenti di Versionamento}: GitHub o BitBucket.
    \item \textbf{Piattaforme}: Windows, Linux, MacOS.
\end{itemize}
\subsubsection*{Aspetti Positivi:}
\begin{itemize}
    \item Progetto ben definito con obiettivi chiari e raggiungibili.
    \item Caso d`uso realistico e stimolante, con dirette applicazioni nel mondo reale.
    \item Necessità di implementare una struttura efficiente vista la mole di dati da
          gestire. Una sfida stimolante per il gruppo.
    \item Disponibilità di contatti con l`azienda proponente per chiarimenti e supporto.
    \item Funzionalità AI per la ricerca semantica come plus interessante ma non come
          focus principale.
    \item Alcune delle tecnologie proposte sono conosciute a parte del gruppo che potrà
          aiutare nella formazione degli altri membri.
\end{itemize}

\subsubsection*{Aspetti Negativi:}
\begin{itemize}
    \item Assenti.
\end{itemize}

\subsubsection*{Possibili Rischi:}
\begin{itemize}
    \item Possibili difficoltà nella gestione della mole di dati e nella loro analisi.
    \item Necessità di bilanciare funzionalità avanzate con la semplicità d`uso
          (interfaccia user-friendly).
\end{itemize}

\subsubsection*{Conclusioni:}
Il capitolato C3 rappresenta una scelta solida e stimolante per il gruppo che potrà affrontare sfide tecniche interessanti e sviluppare competenze rilevanti nel campo della gestione dei dati e dell`analisi documentale.
L`AI svolge un ruolo secondario ma interessante, lasciando spazio al gruppo per esplorare questa tecnologia senza doverla necessariamente padroneggiare in profondità.
L`azienda si è presentata molto disponibile e aperta al dialogo, come dimostrato dall`incontro del 20/10/2025, il che è un ulteriore punto a favore per la scelta di questo capitolato.

\section{Analisi degli Altri Capitolati}

\subsection{Capitolato C1 - Automated EN18031 Compliance Verification}
\subsubsection*{Azienda Proponente:} Bluewind s.r.l.
\subsubsection*{Committente:} Prof. Tullio Verdanega e Prof. Riccardo Cardin.
\subsubsection*{Obiettivo:} Progettare e realizzare un'interfaccia grafica finalizzata alla verifica dell'aderenza dei prodotti alla Direttiva 2014/53/UE (Radio Equipment Directive) in materia di apparecchiature radio.
\subsubsection*{Dominio Applicativo:} Il software è concepito per supportare le aziende nella creazione automatica della documentazione necessaria a dimostrare la conformità dei dispositivi radio alla Direttiva 2014/53/UE.
\subsubsection*{Dominio Tecnico:} Non vengono imposti vincoli nè sul tipo di applicazione (web-based o desktop) nè sulla suddivisione delle parti Backend e Frontend; ma nel caso venisse scelta la suddivisone precedentemente menzionata, viene data preferenza a Python 3.x gestito con un Python Packaging.
\subsubsection*{Aspetti Positivi:}
\begin{itemize}
    \item L`applicazione del progetto in contesti realistici è stimolante.
    \item Libertà nella scelta delle tecnologie da utilizzare.
    \item Il progetto è facile da concepire, cosa che potrebbe facilitarne lo sviluppo.
\end{itemize}

\subsubsection*{Aspetti Negativi:}
Lo sviluppo di un`interfaccia grafica che permetta di scrivere e leggere dai file è una tematica che è già stata tratta nel corso di Programmazione ad Oggetti; per questo motivo lo svolgimento del progetto sarebbe un`attività con limitate possibilità di approfondimento.

\subsubsection*{Possibili Rischi:}
Occorre comprendere come le disposizioni normative possano essere applicate concretamente all`interno del software.

\subsubsection*{Conclusioni:}
Il capitolato C1 risulta interessante per la sua chiarezza e semplicità di realizzazione, però sono proprio questi aspetti che ne limitano la complessità, e quindi anche lo stimolo.

\subsection{Capitolato C2 - Code Guardian}
\subsubsection*{Azienda Proponente:} Var Group S.p.A.
\subsubsection*{Committente:} Prof. Tullio Verdanega e Prof. Riccardo Cardin.
\subsubsection*{Obiettivo:} Il progetto mira a sviluppare una piattaforma web basata su un sistema multi-agente per l'audit e la \textit{remediation} automatica dei repository software, in particolare quelli ospitati su GitHub. Vengono analizzati fattori come:
\begin{itemize}
    \item Test coverage
    \item Aderimento allo standard OWASP
    \item Documentazione
\end{itemize}
Viene proposto alla fine di ogni analisi un report contenente degli indicatori quantitativi riguardo le voci citate sopra. In aggiunta vengono proposte delle "remediation", ossia delle soluzioni per sopperire ad eventuali non conformità della repository.
\subsubsection*{Dominio Applicativo:} Il dominio applicativo si colloca principalmente nell'ambito dello sviluppo software, con collegamenti trasversali ai settori della cybersecurity e dell'intelligenza artificiale. L'obiettivo è promuovere la qualità, la sicurezza e la manutenibilità del codice in contesti collaborativi e distribuiti.
\subsubsection*{Dominio Tecnico:}
\begin{itemize}
    \item \textbf{Backend/Orchestrator}: Node.js, Python
    \item \textbf{Frontend}: React.js
    \item \textbf{Database}: MongoDB o PostgreSQL
    \item \textbf{CI/CD}: GitHub Actions
    \item \textbf{Architettura cloud}: AWS
\end{itemize}
\subsubsection*{Aspetti Positivi:}
\begin{itemize}
    \item Approfondimento del paradigma ad attori proposto nel corso ``Paradigmi di
          Programmazione``.
    \item Design thinking in collaborazione con l'azienda.
    \item Stand up meeting periodici con l`azienda.
    \item Orientamento alla buona qualità del codice, sia come obiettivo di progetto, sia
          come requisiti del software stesso.
    \item Formazione sugli strumenti tecnologici richiesti dai vincoli.
\end{itemize}

\subsubsection*{Aspetti Negativi:}
\begin{itemize}
    \item La curva di apprendimento per il paradigma ad attori è ripida.
    \item Numero di funzionalità potenzialmente elevato, con rischio di ridimensionamento
          del progetto.
\end{itemize}
\subsubsection*{Possibili Rischi:}
\begin{itemize}
    \item Necessità di ridimensionare i requisiti
    \item Possibile complessità nella gestione dell'autenticazione per repository private
    \item Rallentamenti nello sviluppo dovuti alla complessità del paradigma ad attori
\end{itemize}

\subsubsection*{Conclusioni:}
Il capitolato C2 rappresenta un'iniziativa formativa e tecnologicamente avanzata. Il progetto combina aspetti di sviluppo software, cybersecurity e AI, con un forte focus sulla qualità e manutenibilità del codice.
Pur presentando una curva di apprendimento elevata e la possibile necessità di ridimensionare i requisiti, il progetto offre un'importante opportunità di crescita tecnica e professionale. Tuttavia il gruppo, dopo un'attenta discussione, ha deciso di dare priorità ad altri capitolati ritenuti più in linea con le proprie competenze e obiettivi formativi.
\vspace{2.0cm}

\subsection{Capitolato C4 - L'app che Protegge e Trasforma}
\subsubsection*{Azienda Proponente:} Miriade
\subsubsection*{Committente:} Prof. Tullio Verdanega e Prof. Riccardo Cardin.
\subsubsection*{Obiettivo:}
Realizzare un'app mobile multipiattaforma in grado di fornire strumenti di prevenzione, supporto e protezione per le vittime di violenza di genere.
L'app integra funzionalità di analisi comportamentale basate su intelligenza artificiale, sistemi di allerta discreti, accesso a risorse geo-localizzate e sezioni formative.
L'idea finale è quella di offrire un ambiente digitale sicuro, conforme al GDPR, che permetta all'utente di sentirsi tutelato e informato.

\subsubsection*{Dominio Applicativo:}
L'app è pensata per persone a rischio o vittime di violenza di genere e per i centri che offrono supporto.
Mira a fornire strumenti concreti di prevenzione e sicurezza, come il rilevamento automatico di situazioni di pericolo, allarmi silenziosi e una modalità stealth per proteggere l'utente.
Include inoltre un diario criptato, l'accesso ai centri di assistenza, percorsi formativi e una community moderata per condividere esperienze in modo sicuro.

\subsubsection*{Dominio Tecnico:}
\begin{itemize}[leftmargin=*]
    \item \textbf{Frontend}: Flutter.
    \item \textbf{Backend}: AWS Lambda, API Gateway, DynamoDB/RDS, S3, Cognito.
    \item \textbf{AI}: AWS SageMaker o Bedrock per modelli NLP e classificazione del rischio.
    \item \textbf{Architettura}: microservizi o approccio serverless per scalabilità e resilienza.
    \item \textbf{Sicurezza}: crittografia AES-256, autenticazione a più fattori, audit trail, conformità GDPR e accessibilità WCAG 2.1.
\end{itemize}

\subsubsection*{Aspetti Positivi:}
\begin{itemize}[leftmargin=*]
    \item Capitolato con forte impatto sociale e finalità etica rilevante.
    \item Attenzione alla sicurezza e alla privacy dei dati sensibili.
    \item Supporto diretto e affiancamento da parte dell'azienda.
    \item Presenza di linee guida UX orientate all'accessibilità e alla serenità d'uso.
\end{itemize}

\subsubsection*{Aspetti Negativi:}
\begin{itemize}[leftmargin=*]
    \item Numerose funzionalità opzionali che aumentano la complessità progettuale.
    \item Rischi legati all'affidabilità e ai bias dei modelli AI.
    \item Necessità di grande attenzione legale per conformità alle normative su privacy
          e geolocalizzazione.
\end{itemize}

\subsubsection*{Possibili Rischi:}
\begin{itemize}[leftmargin=*]
    \item Fuga di dati sensibili con potenziale esposizione di informazioni o posizioni.
    \item Errori di classificazione AI.
    \item Abuso della community, come uso improprio o divulgazione di dati personali.
    \item Dipendenza da servizi esterni.
    \item Rischi legali a causa di registrazione audio/video e tracciamento GPS soggetti
          a diverse normative.
\end{itemize}

\subsubsection*{Conclusione:}
Il capitolato C4 proposto da Miriade è ambizioso e unisce aspetti tecnici avanzati a un obiettivo sociale significativo.
La realizzazione dell'app comporta sfide importanti, soprattutto in ambito di sicurezza, AI e conformità legale.
Dopo un'attenta discussione, il gruppo ha concluso di non voler proseguire con il capitolato C4.
Pur riconoscendone il valore sociale e l'intento positivo, il progetto risulta estremamente vasto e complesso dal punto di vista tecnico e organizzativo.
La grande quantità di funzionalità previste, unite alla forte dipendenza dall`intelligenza artificiale e all'elevato livello di responsabilità legale e di sicurezza richiesti, rendono il capitolato poco adatto agli obiettivi e alle competenze che il gruppo intende sviluppare in questo percorso.
\vspace{2.0cm}

\subsection{Capitolato C5 - NEXUM}
\subsubsection*{Azienda Proponente:} EGGON
\subsubsection*{Committente:} Prof. Tullio Verdanega e Prof. Riccardo Cardin.
\subsubsection*{Obiettivo:}
Sviluppo di moduli software integrativi al prodotto NEXUM, un software di comunicazione interna e gestione HR, in grado di connettere aziende, collaboratori, dipendenti e Consulenti del Lavoro.
\subsubsection*{Dominio Applicativo:}
Il capitolato si inserisce all`interno dell`applicativo NEXUM di EGGON Srl, una piattaforma digitale che include:
\begin{itemize}
    \item Modulo di messaggistica interna per policy, circolari e notifiche
    \item Modulo di timbratura digitale
    \item Gestione anagrafiche, ruoli e permessi
\end{itemize}
prevedendo perciò di integrare ed evolvere il prodotto con nuove funzionalità da sviluppare, che permettano di introdurre assistenza coadiuvata dall`IA, generativa per creazione di contenuti formali e informali, e in modalità "Co-Pilot" per facilitazione dei rapporti con gli attori esterni coinvolti (consulenti del lavoro). Inoltre è prevista l`estensione delle funzionalità attualmente esistenti per timbratura e gestione/monitoraggio dei turni.

\subsubsection*{Dominio Tecnico:}
\begin{itemize}
    \item \textbf{Database}: Amazon RDS for PostgreSQL
    \item \textbf{Framework Frontend}: Angular per dashboard amministrativa, Next.js per Progressive Web App da destinare agli utenti finali.
    \item \textbf{Backend}: Ruby on Rails, AWS WAF, Sidekiq+SQS
    \item \textbf{Sicurezza}: KMS, Secrets Manager, Amazon Cognito
    \item \textbf{Strumenti di Versionamento}: GitHub.
    \item \textbf{Piattaforme}: Windows, Linux, MacOS.
\end{itemize}
\subsubsection*{Aspetti Positivi:}
\begin{itemize}
    \item Progetto fondato su software esistente, testato e sviluppato secondo stato
          dell`arte.
    \item Utilizzo di strumenti innovativi come l`Intelligenza Artificiale.
    \item Possibilità di partecipare attivamente all`interno dell`ambiente aziendale e
          seguire i processi e le metodologie Agile.
    \item Requisiti chiari e definiti nel dettaglio per ciascun componente del progetto.
\end{itemize}

\subsubsection*{Aspetti Negativi:}
\begin{itemize}
    \item Utilizzo di soluzioni tecniche avanzate non conosciute dal gruppo.
    \item Vincolo di adeguamento e conformità piena ad un`architettura preesistente.
    \item Richiesta di completamento di tutti i moduli come vincolo di accettazione.
\end{itemize}

\subsubsection*{Possibili Rischi:}
\begin{itemize}
    \item Difficoltà ad apprendere e adottare le tecnologie utilizzate.
    \item Disallineamento di disponibilità tra gruppo e azienda alla partecipazione
          completa delle riunioni sprint.
\end{itemize}

\subsubsection*{Conclusioni:}
Il capitolato si presenta interessante dal punto di vista del dominio applicativo e per le modalità di interazione proposte dall`azienda, tuttavia l`inesperienza del gruppo nelle tecnologie adottate unita a richieste implementative potenzialmente complesse hanno portato a non considerarlo abbastanza attrattivo rispetto ad altri capitolati.
\vspace{2.0cm}

\subsection{Capitolato C6 - Second Brain}
\subsubsection*{Azienda Proponente:} Zucchetti S.p.A.
\subsubsection*{Committente:} Prof. Tullio Verdanega e Prof. Riccardo Cardin.
\subsubsection*{Obiettivo:}
Realizzazione di un'applicazione web per la scrittura e la gestione di testi basata su Large Language Models (LLM).
L'obiettivo è creare un “second brain” digitale capace di assistere l'utente nella redazione, revisione, traduzione e critica di testi, sfruttando la potenza dei modelli linguistici per estendere le capacità del classico note-taking.
Il sistema dovrà includere un editor di testo in linguaggio \textit{Markdown} con funzionalità di miglioramento, riassunto, traduzione e analisi critica basate su LLM, integrando inoltre strumenti per la creazione di testi autonomi a partire da prompt forniti dall'utente.

\subsubsection*{Dominio Applicativo:}
Il progetto si colloca nell'ambito della produttività personale e aziendale, in particolare nei contesti di \textit{brainstorming}, redazione di documenti, formazione e creazione di contenuti testuali.
L'applicazione mira a supportare studenti, professionisti e team aziendali nel processo creativo, fungendo da assistente intelligente per la generazione e l'organizzazione delle idee.
Il progetto si ispira a strumenti come \textit{NotebookLM} di Google e \textit{Microsoft Copilot}, con l'intento di offrire una soluzione accessibile e personalizzabile basata su tecnologie open source.

\subsubsection*{Dominio Tecnico:}
\begin{itemize}
    \item \textbf{Frontend:} Applicazione web basata su HTML, CSS e JavaScript, con un'area di editing Markdown e una sezione di rendering in tempo reale.
    \item \textbf{Integrazione LLM:} Accesso a modelli linguistici di grandi dimensioni (es. OpenAI, Gemini, Mistral, Gemma) tramite API conformi allo standard OpenAI.
    \item \textbf{Backend (opzionale):} Implementazione server-side per il salvataggio e la gestione delle note (database SQL o NoSQL), comunicazione via HTTP, potenzialmente in Java o Python.
    \item \textbf{Funzionalità principali:}
          \begin{itemize}
              \item Editing e rendering Markdown.
              \item Riassunto, riscrittura e traduzione del testo.
              \item Critica del testo secondo il modello dei “sei cappelli per pensare” di Edward
                    De Bono.
              \item Generazione di testi da prompt utente (“Distant Writing”).
              \item Salvataggio e recupero delle note.
          \end{itemize}
    \item \textbf{Testing:} Copertura tramite test automatici per PoC, MVP e prodotto finale.
    \item \textbf{Pattern Architetturali:} Progettazione iterativa e modulare per consentire estensioni e revisioni continue.
\end{itemize}

\subsubsection*{Aspetti Positivi:}
\begin{itemize}
    \item Progetto innovativo e in linea con le più recenti evoluzioni nel campo
          dell'intelligenza artificiale.
    \item Ampie possibilità di esplorazione tecnologica e di apprendimento, specialmente
          in ambito LLM.
    \item Requisiti chiari e graduali (PoC, MVP, prodotto finale).
    \item Supporto da parte dell'azienda nello sviluppo e nella verifica del progetto.
\end{itemize}

\subsubsection*{Aspetti Negativi:}
\begin{itemize}
    \item Complessità nell'integrazione delle API LLM e nella gestione dei prompt
          personalizzati.
    \item Potenziali difficoltà legate alla configurazione della comunicazione tra
          frontend e server a causa delle policy di sicurezza web (same-origin).
    \item Elevato livello di sperimentazione e ricerca che può aumentare i tempi di
          sviluppo e test.
\end{itemize}

\subsubsection*{Possibili Rischi:}
\begin{itemize}
    \item Rischio di dipendenza da servizi esterni (API LLM) per il corretto
          funzionamento del sistema.
    \item Difficoltà nella gestione delle revisioni successive e della manutenibilità del
          codice.
    \item Necessità di garantire la privacy e la sicurezza dei dati testuali elaborati.
\end{itemize}

\subsubsection*{Conclusioni:}
Il capitolato C6 proposto da Zucchetti S.p.A. presenta un'elevata innovazione e un forte allineamento con le attuali tendenze dell'intelligenza artificiale generativa, offrendo spunti formativi di grande interesse.
Dopo un'attenta valutazione, tuttavia, il gruppo ha deciso di non selezionare questo capitolato, ritenendolo troppo complesso dal punto di vista tecnico e fortemente dipendente da servizi esterni.
Pur apprezzandone il valore e il potenziale didattico, il team ha preferito orientarsi verso progetti più adeguati alle proprie competenze e risorse disponibili.
\vspace{2.0cm}

\subsection{Capitolato C7 - Sistema di acquisizione dati da sensori}
\subsubsection*{Azienda Proponente:} M31 S.r.l.
\subsubsection*{Committente:} Prof. Tullio Verdanega e Prof. Riccardo Cardin.
\subsubsection*{Obiettivo:}
L'obiettivo del progetto è la realizzazione di un sistema distribuito per l'acquisizione, la gestione e lo smistamento dei dati provenienti da sensori Bluetooth Low Energy (BLE). Tale sistema intende fornire un'infrastruttura scalabile, sicura e multi-tenant (multi utente) capace di raccogliere informazioni da dispositivi differenti tra loro, aggregarle e renderle disponibili tramite una piattaforma cloud centralizzata.
Attraverso l'interazione tra sensori BLE, gateway BLE–WiFi e un ambiente cloud dedicato, il progetto mira a garantire comunicazioni sicure, segregazione dei dati e strumenti avanzati di monitoraggio e visualizzazione. L'obiettivo finale è creare una base solida per applicazioni IoT complesse e realistiche, ponendo le fondamenta per future integrazioni con tecniche di analisi e algoritmi predittivi.
\subsubsection*{Dominio Applicativo:}
Un tale strumento può trovare utilizzo in numerosi ambiti, come l`industria manifatturiera, logistica, healthcare e le smart city. L'Internet of Things (IoT) si basa su tecnologie come il Bluetooth Low Energy (BLE), che consentono la raccolta di dati in modo efficiente e a basso consumo energetico, favorendo la cooperazione in tempo reale tra centinaia di dispositivi.
\subsubsection*{Dominio Tecnico:}
\begin{itemize}
    \item \textbf{Database}: MongoDB per dati non strutturati, PostgreSQL per dati strutturati e Redis (come parte del sistema di caching).
    \item \textbf{Framework Frontend}: Angular, cercando di creare un applicazione mono pagina (SPAs).
    \item \textbf{Framework Backend}: Node.js e Nest.js (con Typescript) per microservizi, Go per la sincronizzazione e NATS o Apache Kafka per la comunicazione tra i microservizi.
    \item \textbf{Infrastuttura}: Google Cloud Platform (GCP) e Kubernetes per l`orchestrazione dei container e microservizi.
    \item \textbf{Piattaforme}: Web
\end{itemize}
\subsubsection*{Aspetti Positivi:}
\begin{itemize}
    \item Progetto caratterizzato da una struttura ben definita con una suddivisione in
          Layer molto chiara.
    \item Casi d'uso realistici e ormai essenziali per il funzionamento della società
          moderna.
    \item Utilizzo di molte tecnologie con campi d`uso eterogenei tra loro (parzialmente
          studiate da parte del gruppo).
    \item Libertà nella selezione delle tecnologie da adottare, previa motivazione e
          approvazione da parte di M31.
\end{itemize}

\subsubsection*{Aspetti Negativi:}
\begin{itemize}
    \item Si ha un`elevata complessità tecnica data dalle numerose tecnologie da
          utilizzare.
    \item Infrastruttura piuttosto complessa da configurare e gestire con conseguente
          lungo periodo di studio preliminare.
\end{itemize}

\subsubsection*{Possibili Rischi:}
\begin{itemize}
    \item Possibili difficoltà nell`integrazione tra le diverse componenti del sistema.
\end{itemize}

\subsubsection*{Conclusioni:}
Il capitolato C7 rappresenta un'opportunità stimolante per il gruppo, offrendo la possibilità di affrontare sfide legate a sistemi distribuiti e gestione sicura dei dati IoT.
Il progetto consente di esplorare tecnologie moderne e richieste dal mercato, come microservizi, cloud e interfacce web. Nel complesso, C7 offre un percorso formativo completo e motivante, con ricadute concrete sulle competenze tecniche del gruppo, utili nel mondo del lavoro.
\vspace{2.0cm}

\subsection{Capitolato C8 - SmartOrder}
\subsubsection*{Azienda Proponente:} ERGON Informatica S.R.L.
\subsubsection*{Committente:} Prof. Tullio Verdanega e Prof. Riccardo Cardin.
\subsubsection*{Obiettivo:} Sviluppo di un sistema di analisi di contenuti multimodali per la generazione automatica di ordini da inserire all`interno di sistemi ERP grazie all`ausilio di LLM. Tale sistema riuscirebbe a garantire l`automazione del sistema di ordine, facilitandone l`avanzamento di richieste lato cliente.
\subsubsection*{Dominio Applicativo:} Servizi di assistenza alla vendita. Con un sistema come SmartOrder sarebbe possibile velocizzare la creazione di ordini, semplificandone il processo di acquisizione, in quanto si presuppone che il sistema conosca il contesto di vendita e sia in grado di interpretare richieste complesse.
\subsubsection*{Dominio Tecnico:}
\begin{itemize}
    \item \textbf{Database}: A libera scelta, per esempio SQL Server Express, MySql o MariaDB
    \item \textbf{Framework Frontend}: Angular, React o .NET Blazor, anche in questo caso la scelta è libera.
    \item \textbf{Strumenti AI}: \begin{itemize}
              \item BERT, RoBERTa o GTP di OpenAi per NLP
              \item Tesseract OCR, EasyOCR, Reti Convoluzionari (CNN) o Vision Transformer per
                    Riconoscimento Ottico dei Caratteri
              \item Whisper di OpenAi o Google Speech-to-Text per il riconoscimento vocale
          \end{itemize}
    \item \textbf{Comunicazione LLM-UI}: API REST
    \item \textbf{Comunicazione LLM-Database}: Connessione standard con connettori da fonti ODBC oppure tramite l`implementazione di un middleware ad hoc
\end{itemize}
\subsubsection*{Aspetti Positivi:}
\begin{itemize}
    \item Progetto improntato sulle tecnologie AI più recenti e interessanti.
    \item Utilizzo dell`AI non banale, con un chiaro valore aggiunto per l`utente finale.
    \item Caso d`uso realistico e stimolante.
    \item L`azienda è certificata dallo standard ISO:9001, il che garantisce una certa
          qualità nei processi di sviluppo.
    \item Esperienza pregressa con università e studenti.
\end{itemize}

\subsubsection*{Aspetti Negativi:}
\begin{itemize}
    \item La mole di lavoro richiesta potrebbe essere elevata, vista la complessità del
          dominio applicativo.
    \item Presentazione del capitolato non del tutto chiara anche in sede di incontro con
          l`azienda.
    \item Necessità di apprendere e padroneggiare diverse tecnologie AI, che potrebbero
          risultare troppo complesse per il gruppo.
    \item Centralità dell`AI nel progetto, che potrebbe risultare un rischio se non
          adeguatamente gestita.
\end{itemize}

\subsubsection*{Possibili Rischi:}
\begin{itemize}
    \item Complessità degli argomenti AI proposti che potrebbero risultare troppo
          difficili da padroneggiare.
    \item Rischio di non riuscire a integrare correttamente le diverse tecnologie AI.
\end{itemize}

\subsubsection*{Conclusioni:}
Il capitolato C8 rappresenta una scelta ambiziosa per il gruppo, che potrebbe affrontare sfide tecniche significative e sviluppare competenze avanzate nel campo dell`intelligenza artificiale applicata.
Tale mole di lavoro e tecnologie richieste potrebbero però risultare eccessive per il gruppo, soprattutto considerando la necessità di padroneggiare diverse tecnologie AI complesse come il \textit{"fine tuning"} richiesto per fornire al sistema di LLM il contesto aziendale.
Seppur l`azienda abbia esperienza con studenti e università e potrebbe fornire supporto anche organizzativo vista la certificazione ISO:9001, l`incontro non è riuscito a chiarire tutti i dubbi del gruppo, soprattutto riguardo alla complessità tecnologica.
\vspace{2.0cm}

\subsection{Capitolato C9 – View4Life}

\subsubsection*{Azienda Proponente}
Vimar S.p.A.

\subsubsection*{Committente}
Prof. Tullio Verdanega e Prof. Riccardo Cardin.

\subsubsection*{Obiettivo}
Progettare e realizzare una piattaforma unica per la gestione completa degli impianti Smart nelle residenze protette per anziani. La piattaforma deve:
\begin{itemize}
    \item Interfacciarsi con dispositivi Vimar View Wireless;
    \item Fornire un applicativo web responsive per il personale sanitario;
    \item Realizzare un'infrastruttura Cloud per ospitare tutte le funzioni
          dell'applicativo;
    \item Gestire allarmi, statistiche energetiche e analytics avanzate.
\end{itemize}

\subsubsection*{Dominio Applicativo}
Il progetto si colloca nel settore della domotica avanzata, con il focus sulla gestione di residenze protette per anziani. Gli utenti principali della piattaforma sono il personale sanitario, gli amministratori delle residenze e gli utenti finali. Le funzionalità chiave previste includono la sicurezza degli ambienti, il controllo delle condizioni ambientali come illuminazione e temperatura, la gestione centralizzata degli allarmi e il monitoraggio energetico per ottimizzare i consumi.

\subsubsection*{Dominio Tecnico:}
\begin{itemize}
    \item \textbf{Front-end}: Applicativo web responsive, librerie JS/CSS, design wireframe tramite Figma/Draw.io.
    \item \textbf{Back-end}: Node.js con Express, Java con Spring, Python con Flask o FastAPI. SDK per interfaccia KNX IoT 3rd-party API.
    \item \textbf{Infrastruttura Cloud}: Container Docker, Infrastructure as Code, servizi opzionali AWS.
    \item \textbf{Database}: Relazionali per applicativo web, NoSQL per analytics.
    \item \textbf{Autenticazione e API}: KNX IoT 3rd-party API, OAuth2, push notification tramite subscription KNX IoT.
    \item \textbf{Versionamento}: Git, repository pubblico, licenza open source.
\end{itemize}

\subsubsection*{Aspetti Positivi:}
\begin{itemize}
    \item Soluzione completa e scalabile per la gestione di residenze protette.
    \item Integrazione avanzata con dispositivi domotici e servizi Cloud.
    \item Utilizzo di tecnologie moderne e standardizzate.
    \item Opportunità di creare SDK riutilizzabile per integrazione da terze parti.
\end{itemize}

\subsubsection*{Aspetti Negativi:}
\begin{itemize}
    \item Complessità tecnica elevata per gestione di Cloud, container e API KNX IoT.
    \item Test obbligatori con coverage elevata.
    \item Alcune funzionalità opzionali richiedono implementazioni aggiuntive.
\end{itemize}

\subsubsection*{Possibili Rischi:}
\begin{itemize}
    \item Malfunzionamenti nella comunicazione tra dispositivi e Cloud.
    \item Problemi di sicurezza e privacy dei dati sensibili degli anziani.
    \item Ritardi nello sviluppo dovuti alla complessità tecnologica.
\end{itemize}

\subsubsection*{Conclusioni:}
Il progetto View4Life rappresenta un esempio avanzato di applicazione della domotica in contesti residenziali. Tuttavia, il gruppo ha deciso di scartare questo capitolato a causa dell'elevata complessità tecnica richiesta e dell'integrazione con dispositivi IoT e servizi esterni. Il progetto è stato ritenuto meno adatto alle competenze attuali del gruppo e più rischioso in termini di tempistiche e mantenimento.
Follia-baluarda
\vspace{0.5cm}

\vfill
\begin{flushright}
    \textit{7-ZPUs}
\end{flushright}

\end{document}